\title{flexmem: Allowing R to Natively Support Out-of-core Data Structures}
\author{by Michael Kane, and Bryan Lewis}

\maketitle

\abstract{
An abstract of less than 150 words.
}

%Introductory section which may include references in parentheses, say
%\citep{R:Ihaka+Gentleman:1996} or cite a reference such as
%\citet{R:Ihaka+Gentleman:1996} in the text.

%\section{Introduction}

Challenges to supporting big-data in R
\begin{enumerate}
  \item indexing: fixed in the next version of R
  \item out-of-core data structures: memory-map - integration is a challenge
\end{enumerate}

\section{Past work}

Past approaches
\begin{enumerate}
  \item packages: bigmemory and ff
  \item doppleganger
\end{enumerate}

%This section may contain a figure such as Figure \ref{figure:onecolfig}.

%\begin{figure}
%\vspace*{.1in}
%\framebox[\textwidth]{\hfill \raisebox{-.45in}{\rule{0in}{1in}}
%                      A picture goes here \hfill}
%\caption{\label{figure:onecolfig}
%A normal figure only occupies one column.}
%\end{figure}

\section{Overriding R's default memory allocator}

Most software applications, including R, make use of dynamic
libraries. These provide standard functionality, such as math operations,
graphing capabilities, and printing to the console, freeing software developers 
from their reimplementation. Applications make use of these resources through
a process called dynamic linking.  The operating
system starts an application by loading both its machine code along with
all of the dynamically linked libraries on which the application depends and 
starts the application.

Unix-like operating systems, like Linux and Mac OS X, allow a developer to 
specify new functionality for standard functions. This is done by creating
a new dynamically linked library (DLL), which defines the new functionality. 
This new library is then ``preloaded'' by setting an appropriate environment 
variable to its location. For Linux systems this is the \code{LD\_PRELOAD} 
variable and on Mac OS X it is \code{DYLD\_LIBRARY\_PATH}. 

After the new DLL is created and the preload environment variable is set an 
application is started. Standard libraries are loaded first, then the preloaded
libraries. This allows the preloaded library to make use of standard 
functionality if needed. The application then starts as usual. New functionality
is automatically available to the application user.

Among the functionalities included by dynamic libraries is the ability
to dynamically allocate and free available memory. All programs that create
new data structures during the execution of an application rely on this facility
to allocate space to hold these data structures. The standard dynamic memory
allocation functions used in R are \code{malloc}, \code{calloc}, 
\code{realloc}, and \code{free}. These
functions are provided by stdlib, the standard C library. These are also
the functions that are overridden in the \pkg{flexmem} package. 
Memory allocation
calls are intercepted by the flexmem DLL. When the amount of memory to allocate
exceeds a threshold, a memory-mapped file is created to hold the data
structure. The address of the memory-mapped vector is stored dictionary, 
implemented by the Standard Template Library \citep{Plauger:STL} map data 
structure.  This new memory is returned to R to create a standard \code{SEXP}. 
The R interpreter handles this memory exactly the same as when the memory 
is created in RAM. 

When the storage needed to hold a data structure is no longer needed, it
is de-allocated with the \code{free} function. Once again, this function call
once again goes to the \pkg{flexmem} library. The dictionary checks to see if
the address to be freed is memory-mapped. If it is than the resource is
freed and the file holding the data are erased. If it is not and the 
data was allocated from RAM then the standard free function is called.
Both the allocation and de-allocation changes are transparent to both the 
R interpreter and the R user. The user receives the benefits of being able
to allocate data structures that are limited in size by hard drive space,
not RAM. And R, along with its associated packages is able to make use
of this new functionality with no changes whatsoever.

<GIVE INSTRUCTIONS FOR USING FLEXMEM AND VERIFYING THAT IT WORKS HERE>

\section{Managing implicit copying with R5 reference objects}

Overriding the memory allocator solves the problem of creating native
R objects that may be larger than available RAM. However, care needs
be taken when computing with these potentially massive objects, 
especially when they can be implicitly copied. While extra copies will
not cause R to run out of memory per se, it can lead to programs running
slowly when the data are large and copies are made often.

This challenge can easily be addressed though by wrapping objects in 
R5 reference classes. This class system was made available in 2011 and
allows developer to create objects that exhibit reference semantics.
That is, they must be explicitly copied and simple assignments to 
reference objects result in new R objects that point to the same 
underlying data structure. The \pkg{flexmem} package provides R5 objects 
to wrap native objects allowing developers to quickly and easily create
reference objects based on non-reference objects. Since \pkg{flexmem} 
was intended for the various matrix types provided by R, common matrix
functionality is also provided by these wrapper-reference objects.

\section{Parallel programming considerations}

\begin{enumerate}
\item multicore will ``just work''
\item other parallel packages may require too much communication overhead
for big data structures.
\item Attaching to an existing file. An equivalent to R's load.
<BRYAN> Can we use create an SEXP, swap the addresses in the interpreter,
and free the SEXP R created?
\end{enumerate}

\section{Platform limitations}

<BRYAN> Windows - how can we get this to work?

\section{Future work}

C++ support for new and delete

\section{Conclusion}

Don't change R, override memory allocation is a viable approach to big-data 
computing in R.

%There will likely be several sections, perhaps including code snippets, such
%as
%\begin{example}
%  x <- 1:10
%  result <- myFunction(x)
%\end{example}

%\section{Summary}
%
%This file is only a basic article template. For full details of \emph{The R Journal}
%style and information on how to prepare your article for submission, see the
%\href{http://journal.r-project.org/latex/RJauthorguide.pdf}{Instructions for Authors}.

%\bibliography{example}

\begin{thebibliography}{1}
\expandafter\ifx\csname natexlab\endcsname\relax\def\natexlab#1{#1}\fi
\expandafter\ifx\csname url\endcsname\relax
  \def\url#1{{\tt #1}}\fi

\bibitem[Adler et al.(2012)]{ff:Adler+et al.:2012}
A.~D and C.~Glaser and O.~Nenadic and J. Oehlschl\"agel and W.~Zucchini.
\newblock {\em ff: memory-efficient storage of large data on disk and fast access functions}, 2012.
\newblock URL \url{http:CRAN.R-project.org/package=ff}.
\newblock R package version 2.2-10.

%\bibitem[Ihaka and Gentleman(1996)]{R:Ihaka+Gentleman:1996}
%R.~Ihaka and R.~Gentleman.
%\newblock R: A language for data analysis and graphics.
%\newblock {\em Journal of Computational and Graphical Statistics}, 5\penalty0
%  (3):\penalty0 299--314, 1996.
%\newblock URL \url{http://www.amstat.org/publications/jcgs/}.

\bibitem[Kane and Emerson(2012)]{bigmemory:Kane+Emerson:2012}
M.~Kane and J.~Emerson.
\newblock {\em bigmemory: Manage massive matrices with shared memory and memory-mapped files}, 2012.
\newblock URL \url{http://www.bigmemory.org}.
\newblock R package version 4.3.0.

\bibitem[Kane et al.(2012)]{ScalableStrategies:Kane+et al.:2012}
M.~Kane and J.~Emerson and S.~Weston.
\newblock Scalable Strategies for Computing with Massive Data.
\newblock {\em Preprint}, 2012.

\bibitem[Plauger(1995)]{Plauger:STL}
P.~J.~Plauger.
\newblock {The Standard Template Library}.
\newblock {\em C/C++ Users Journal}, {\bf 13}(12), 10--20, 1995.

\end{thebibliography}

\address{Michael Kane\\
  Yale University\\
  300 George Street\\
  Suite 555\\
  New Haven, CT 06511\\
  USA}\\
\email{michael.kane@yale.edu}

\address{Bryan Lewis\\
  Paradigm4\\
  Address\\
  Country}\\
\email{blewis@illposed.net}

